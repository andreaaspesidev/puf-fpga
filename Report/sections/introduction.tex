% Versione Vivado, board e chip 
% che avete preso come target, frequenza operativa, 
% eventuali software aggiuntivi usati.
% Articolo di riferimento implementazione

A wide variety of Physical Unclonable Functions have been proposed in literature, exploiting
some random and unpredictable properties of electronic chips to generate unique and reliable
responses required by specific application like chip identification, cryptographic key generation, logic obfuscation
and IP protection. Among the ones which are suitable to be implemented on FPGA, Transient Effect Ring Oscillator (TERO) 
PUFs have been proved to have very good properties in terms of uniqueness, reliability and resistance to
environmental changes.\\\\
In \cite{ref_pap} by A.Wild et al., an in-depth analysis of RO, Loop and TERO PUFs has been performed comparing
the responses coming from 100 Basys3 boards on which 1280 PUFs were implemented.
Our project consists in a further analysis of the data acquired by the authors of the paper, followed by a SystemVerilog
implementation of the \textbf{TERO PUF for device identification purposes}, targeting both Basys3 and CmodA7-35t boards 
(both mounting XC7-A35T FPGA), tested at 100Mhz. \\\\
In order to design and test the PUF modules Xilinx Vivado 2020.2 was used. The analysis of the PUF responses and
the generation of the IDs has been performed through some dedicated Python scripts and jupyter notebooks.\\
