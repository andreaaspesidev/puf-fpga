Though the composition and placement of the TERO loops are the same compared to 
the Reference paper \cite{ref_pap}, other design choices differ.
In the original implementation, some parts of the design were missing.

\subsection{Averaging of TERO loop oscillations}
We implemented our version of the averaging step in the evaluation cycle, 
since it was missing in the original implementation code.
Multiple measurements of each loop's oscillations are added using the same shared counter 
(to optimize resources), with a number of bits extended for this purpose. 
At the end, a shift is performed to get the averaged measurement.

\subsection{Finite State Machines}
During our revision of the reference implementation, we found zero to little comments and 
lots of unused code.
We decided to write from scratch the two FSM (both present also in the original implementation),
and changed completely the one dedicated to the communication.

\subsection{Routing}
Most of the routes were conserved, we rewrote the script in a more readable way, and fixed some
warnings about combinatorial loops that prevented the generation of the Bitstream 
in newer versions of Vivado.

\subsection{Memorizing oscillations}
We used a Xilinx FIFO generator IP, in First-Word-Falls-Through configuration to generate the FIFO
that memorizes the final measurements of each TERO loops, and converts the width of each
measurement in single byte words, ready to be sent without further transformations via UART.
The empty FIFO signal was used instead of a dedicated counter to detect the end of the data transfer.
Some internal registers used to buffer signals were removed because not needed for the TERO.